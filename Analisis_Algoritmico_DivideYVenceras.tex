\documentclass{article}
\usepackage[utf8]{inputenc}
\usepackage[normalem]{ulem}
\usepackage{verbatim}
\usepackage[spanish]{babel}
\usepackage{color}
\usepackage{xcolor}
\usepackage[margin=2.5cm]{geometry}
\usepackage{kantlipsum}
\usepackage{hyperref}
\usepackage{graphicx}
\usepackage{listings}
\usepackage{import}
\usepackage{fancyhdr}
\usepackage{tocloft}
\usepackage{adjustbox}
\usepackage{svg}
\usepackage{titling}

\definecolor{mGreen}{rgb}{0,0.6,0}
\definecolor{mGray}{rgb}{0.5,0.5,0.5}
\definecolor{backgroundColour}{rgb}{0.95,0.95,0.92}

\lstdefinestyle{CStyle}{
    backgroundcolor=\color{backgroundColour},   
    commentstyle=\color{blue},
    keywordstyle=\color{red},
    numberstyle=\tiny\color{mGray},
    stringstyle=\color{mGreen},
    basicstyle=\footnotesize,
    breakatwhitespace=false,         
    breaklines=true,                 
    captionpos=b,                    
    keepspaces=true,                 
    numbers=left,                    
    numbersep=5pt,                  
    showspaces=false,                
    showstringspaces=false,
    showtabs=false,                  
    tabsize=2,
    language=C
}

\pagestyle{fancy}
\lhead{Tecnologías específicas de la ingeniería informática}
\rhead{3º Grado de Ingeniería Informática}
\chead{}
\cfoot{\thepage}
\renewcommand{\headrulewidth}{0.2pt}
\renewcommand{\footrulewidth}{0.2pt}
\renewcommand\cftsecleader{\cftdotfill{\cftdotsep}}

\font\fuente=cmr12 at 40pt

\renewcommand\maketitlehooka{\null\mbox{}\vfill}
\renewcommand\maketitlehookd{\vfill\null}

\title{\fuente Análisis algorítmico: Divide y vencerás}
\author{Siro Sánchez del Amo}
\date{Febrero 2018}

\begin{document}
\begin{titlingpage}
\vfill 
\maketitle

\par
\raisebox{-.5\height}{\includegraphics[width=4cm]{logo-facultad-informatica-500-300x114.jpg}}%
\hfill
\raisebox{-.5\height}{\includegraphics[width=4cm]{umulogo.jpg}}%
\par

\thispagestyle{empty}
\end{titlingpage}
\newpage

\tableofcontents
\thispagestyle{fancy}
\newpage

\section{Introducción}
Este documento consiste en un análisis algorítmico de la técnicas ''Multiplicación directa'', ''Divide y vencerás'' y ''Karatsuba y Ofman''. Toda teoría, conocimiento adquirido y algoritmos vienen dados en el libro de la asignatura de AED2 \cite{AED2}

\section{Algoritmo para ''Enteros largos''}
\subsection{Definición del problema}
El problema que se plantea es la multiplicación de números muy largos, lo cual a la larga puede dar problemas ya sea por llegar al límite o por lentitud de las operaciones, para poder realizar operaciones muy largas hemos usado listas y en cada posición de estas habrá un caracter, esto nos dará mas juego que usando simplemente enteros o doble precisión. Con esto en mano, procederemos al análisis de los algoritmos que hemos adaptado al problema.

\subsection{Solución al problema}
\subsubsection{Multiplicación directa}
La multiplicación directa fue programada con el siguiente código. Las funciones elementales se han omitido.
\begin{lstlisting}[style=CStyle]
list<char> multiplicacionDirecta(list<char> EL1,list<char> EL2){
	list<char> resultado;
	list<char> auxiliar;
	int negativo=0;
	list<char>::iterator it1=EL1.begin();
	list<char>::iterator it2=EL2.begin();
	if((*it1)=='-') {
		EL1.erase(it1);
		negativo++;
	}
	if((*it2)=='-') {
		EL2.erase(it2);
		negativo++;
	}
	it2=EL2.end();
	it2--;
	int desplazamiento=1;
	resultado=multiplicacionSimple(EL1,(*it2));
	for(unsigned int i=0;i<EL2.size()-1;i++){
		it2--;
		auxiliar=multiplicacionSimple(EL1,(*it2));
		for(int j=0;j<desplazamiento;j++) auxiliar.insert(auxiliar.end(),'0');
		resultado=suma(resultado,auxiliar);
		desplazamiento++;
	}
	list<char> uno;
	uno.insert(uno.begin(),'1');
	if(negativo==1 && !menor(resultado,uno)) resultado.insert(resultado.begin(),'-');
	return resultado;
}
\end{lstlisting}
\newpage

\subsubsection{Divide y vencerás}
Divide y vencerás fue programado con el siguiente código. Las funciones elementales se han omitido.
\begin{lstlisting}[style=CStyle]
list<char> DYV(list<char> EL1,list<char> EL2,int n,int base){
	list<char> resultado;
	if(n==base) return multiplicacionDirecta(EL1,EL2);
	else{
		list<char> w = divisionSuperior(EL1);
		list<char> y = divisionSuperior(EL2);
		list<char> x = divisionInferior(EL1);
		list<char> z = divisionInferior(EL2);
		list<char> m1 = DYV(w,y,n/2, base);
		list<char> m2 = DYV(w,z,n/2, base);
		list<char> m3 = DYV(x,y,n/2, base);
		list<char> m4 = DYV(x,z,n/2, base);
		resultado=suma(desplazar(m1,n),suma(desplazar(suma(m2,m3),n/2),m4));
		return resultado;
	}
}
\end{lstlisting}

\subsubsection{Karatsuba y Ofman}
Karatsuba fue programado con el siguiente código. Las funciones elementales se han omitido.
\begin{lstlisting}[style=CStyle]
list<char> KYO(list<char> EL1, list<char> EL2, int n, int base){
	list<char> resultado;
	if (n==base) return multiplicacionDirecta(EL1,EL2);
	else{
		list<char> w = divisionSuperior(EL1);
		list<char> y = divisionSuperior(EL2);
		list<char> x = divisionInferior(EL1);
		list<char> z = divisionInferior(EL2);
		list<char> m1 = KYO(w,y,n/2, base);
		list<char> m2 = KYO(x,z,n/2, base);
		list<char> m3 = KYO(restar(w,x),restar(z,y),n/2,base);
		resultado=suma(desplazar(m1,n),suma(desplazar(suma(suma(m3,m2),m1),n/2),m2));
		return resultado;
	};
}
\end{lstlisting}
\newpage

\section{Análisis de los algoritmos}
\subsection{Multiplicación directa}
Calculamos la cota inferior:
$$t_m(n)=14+4+5n[Mult Simple]+\sum_{i=0}^{n-1} (2+4+5 n[MultSimple]+12+4 n[Suma]+ \sum_{j=0}^{m} (2))$$
$$t_m(n)=18+5n+18n-18+9n^2-9n+2mn-2m$$
$$t_m(n)=9n^2-14n+2mn-2m$$
$$t_m(n)\in\Omega(n^2)$$
\\
Calculamos cota superior:
$$t_M(n)=5+6+3+7+5n[MultSimple]+ \sum_{i=0}^{n-1}(2+7n+5[MultSimple]+\sum_{j=0}^{m}(2)+17+8n[Suma]+1)+4$$
$$t_M(n)=25+5n-2n-2+7n^27n+5n-5+2mn-2m+17n-17+8n^2-8n+n-1$$
$$t_M(n)=15n^2+15n+2mn-2m$$
$$t_M(n)\in O(n^2)$$
\\
Lo que al final deriva en:
$$
	\left.
        \begin{array}{ll}
           	t_M(n)\in O(n^2) \\
			t_m(n)\in\Omega(n^2)
        \end{array}
    \right\} t_p(n)\in \theta(n^2).
$$
\newpage
\begin{center}
En la siguiente tabla se pueden ver los distintos experimentos realizados para este algoritmo: 
\end{center}
\label{md-table}
\begin{adjustbox}{width=0.5\textwidth,center=\textwidth}
\begin{tabular}{|l|l|l|}
\hline
{Dígitos} 					   & {T.experimental} 					   & {Memoria} 						\\ \hline
1                              & $\sim$0.00025s                        & Less than 1KB                  \\ \hline
2                              & $\sim$0.00031s                        & Less than 1KB                  \\ \hline
4                              & $\sim$0.00038s                        & Less than 1KB                  \\ \hline
8                              & $\sim$0.00049s                        & Less than 1KB                  \\ \hline
16                             & $\sim$0.00125s                        & Less than 1KB                  \\ \hline
32                             & $\sim$0.00450s                        & Less than 1KB                  \\ \hline
64                             & $\sim$0.01426s                        & Less than 1KB                  \\ \hline
128                            & $\sim$0.05532s                        & Less than 1KB                  \\ \hline
256                            & $\sim$0.14827s                        & $\sim$33KB                     \\ \hline
512                            & $\sim$0.36122s                        & $\sim$66KB                     \\ \hline
1024                           & $\sim$1.22063s                        & $\sim$132KB                    \\ \hline
\end{tabular}
\end{adjustbox}

\newpage
\subsection{Divide y vencerás}
La función \texttt{DYV} es una función recursiva, así pues se calculará de manera distinta:
$$
	t(n) = \left\{
        \begin{array}{ll}
           	t(n)=n^2 & n=base \\
			t(n)=8n+4t(\frac{n}{2}) & n>base
        \end{array}
    \right.
$$

$$n = 2^k \rightarrow t(2^k)-4t (2^{k-1})=0$$
$$\frac{x^{2^k}-4x^{2^{k-1}}}{x^{2^{k-1}}} = x-4 $$
$$ (x-4)=8 \times 2^k $$
$$ (x-4)=(x-2) $$
$$ c_1\times 2^k+c_2 \times 4^k=0 $$
$$ c_1 \times 2^{\log n}+c_2 \times 4^{\log n} $$
$$ c_1 \times n^{log 2}+c_2 \times n^{log 4} $$
$$ c_1 \times n+c_2 \times n^2 $$
$$ t(n)\in O(n^2) $$

\begin{center}
En la siguiente tabla se pueden ver los distintos experimentos realizados para este algoritmo: 
\end{center}
\label{dyv-table}
\begin{adjustbox}{width=0.5\textwidth,center=\textwidth}
\begin{tabular}{|l|l|l|}
\hline
{Dígitos} 					   & {T.experimental} 					   	& {Memoria} 					 \\ \hline
1                              & $\sim$0.00002s                         & Less than 1KB                  \\ \hline
2                              & $\sim$0.00005s                        	& Less than 1KB                  \\ \hline
4                              & $\sim$0.00012s                        	& Less than 1KB                  \\ \hline
8                              & $\sim$0.00051s                        	& Less than 1KB                  \\ \hline
16                             & $\sim$0.00481s                        	& Less than 1KB                  \\ \hline
32                             & $\sim$0.01741s                       	& Less than 1KB                  \\ \hline
64                             & $\sim$0.02497s                        	& Less than 1KB                  \\ \hline
128                            & $\sim$0.05821s                        	& Less than 1KB                  \\ \hline
256                            & $\sim$0.08921s                       	& $\sim$79KB                     \\ \hline
512                            & $\sim$0.16236s                       	& $\sim$104KB                    \\ \hline
1024                           & $\sim$1.18317s                        	& $\sim$200KB                    \\ \hline
\end{tabular}
\end{adjustbox}

\newpage
\subsection{Karatsuba y Ofman}
La función \texttt{KYO} es una función recursiva, así pues se calculará de manera distinta:
$$
	t(n) = \left\{
        \begin{array}{ll}
           	t(n)=n^2 & n=base \\
			t(n)=10n+3t(\frac{n}{2}) & n>base
        \end{array}
    \right.
$$
$$n = 2^k \rightarrow t(2^k)-3t(2^{k-1})=0$$
$$\frac{x^{2^k}-3x^{2^{k-1}}}{x^{2^{k-1}}} = x-3 $$
$$ (x-3)=10 \times 2^k $$
$$ (x-3)=(x-2) $$
$$ c_1\times 2^k+c_2 \times 3^k=0 $$
$$ c_1 \times 2^{\log n}+c_2 \times 3^{\log n} $$
$$ c_1 \times n^{log 2}+c_2 \times n^{log 3} $$
$$ c_1 \times n+c_2 \times n^{1.59} $$
$$ t(n)\in O(n^{1.59}) $$

\begin{center}
En la siguiente tabla se pueden ver los distintos experimentos realizados para este algoritmo: 
\end{center}
\label{kyo-table}
\begin{adjustbox}{width=0.5\textwidth,center=\textwidth}
\begin{tabular}{|l|l|l|}
\hline
{Dígitos} 					   & {T.experimental} 					   	& {Memoria} 					 \\ \hline
1                              & $\sim$0.00002s                        	& Less than 1KB                  \\ \hline
2                              & $\sim$0.00012s                        	& Less than 1KB                  \\ \hline
4                              & $\sim$0.00049s                        	& Less than 1KB                  \\ \hline
8                              & $\sim$0.00165s                        	& Less than 1KB                  \\ \hline
16                             & $\sim$0.00422s                        	& Less than 1KB                  \\ \hline
32                             & $\sim$0.00826s                       	& Less than 1KB                  \\ \hline
64                             & $\sim$0.01724s                      	& Less than 1KB                  \\ \hline
128                            & $\sim$0.02942s                        	& Less than 1KB                  \\ \hline
256                            & $\sim$0.08425s                      	& Less than 10KB                 \\ \hline
512                            & $\sim$0.09029s                      	& Less than 20KB                 \\ \hline
1024                           & $\sim$1.09802s                        	& Less than 30KB                 \\ \hline
\end{tabular}
\end{adjustbox}

\newpage
\section{Conclusiones experimentales}

\begin{figure}[htb]
\centering{\scalebox{.5}{\input{grafica.pdf_tex}}}
\caption{Gráfica de los tiempos de ejecución}
\end{figure}

Como conclusiones podemos sacar varias:
	\begin{itemize}
	\item El algoritmo preferido es sin duda \texttt{Karatsuba}. Al principio no se diferencia mucho de \texttt{Multiplicación directa} y \texttt{Divide y vencerás} pero conforme van aumentando la cantidad de números a multiplicar se nota.
	\item En cuanto a la partición base de los algoritmos, cuanto menos base, más recursividad tendrá que realizar, eso quiere decir más tiempo. Entonces si sólo realizáramos una única recursividad y las particiones por método directo, conseguiremos tiempos muy rápidos.
	\end{itemize}
	
\newpage

\bibliographystyle{plain}
\bibliography{ReferenciasTex} 

\end{document}
