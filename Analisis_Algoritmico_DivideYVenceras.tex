\documentclass{article}
\usepackage[utf8]{inputenc}
\usepackage[normalem]{ulem}
\usepackage{verbatim}
\usepackage[spanish]{babel}
\usepackage{color}
\usepackage[margin=2.5cm]{geometry}
\usepackage{kantlipsum}
\usepackage{hyperref}
\usepackage{graphicx}
\usepackage{listings}
\usepackage{import}
\usepackage{fancyhdr}
\usepackage{tocloft}

\hypersetup{
    colorlinks=true,
    linktoc=page,
    linkcolor=red,
}

\pagestyle{fancy}
\lhead{Tecnologías específicas de la ingeniería informática}
\chead{}
\cfoot{\thepage}
\rhead{}
\renewcommand{\headrulewidth}{0.2pt}
\renewcommand{\footrulewidth}{0.2pt}
\renewcommand\cftsecleader{\cftdotfill{\cftdotsep}}

\newcommand{\bold}[1]{\textbf{#1}}

\title{Análisis algorítmico: Divide y vencerás}
\author{Siro Sánchez del Amo}
\date{Febrero 2018}

\begin{document}
\maketitle

\thispagestyle{empty}
\newpage

\tableofcontents
\thispagestyle{fancy}
\newpage

\section{Introducción}
Este documento consiste en un análisis algorítmico de la técnicas ''Multiplicación directa'', ''Divide y vencerás'' y ''Karatsuba y Ofman''.

\section{Algoritmo para ''Enteros largos''}
\subsection{Definición del problema}
El problema que se plantea es la multiplicación de números muy largos, lo cual a la larga puede dar problemas ya sea por llegar al límite o por lentitud de las operaciones, para poder realizar operaciones muy largas hemos usado listas y en cada posición de estas habrá un caracter, esto nos dará mas juego que usando simplemente enteros o doble precisión. Con esto en mano, procederemos al análisis de los algoritmos que hemos adaptado al problema.

\subsection{Solución al problema}

\section{Análisis de los algoritmos}
\subsection{Multiplicación directa}

\subsection{Divide y vencerás}

\subsection{Karatsuba y Ofman}

\section{Conclusiones experimentales}
\end{document}
