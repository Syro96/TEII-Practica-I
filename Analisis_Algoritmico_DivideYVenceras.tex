\documentclass{article}
\usepackage[utf8]{inputenc}
\usepackage[normalem]{ulem}
\usepackage{verbatim}
\usepackage[spanish]{babel}
\usepackage{color}
\usepackage[margin=2.5cm]{geometry}
\usepackage{kantlipsum}
\usepackage{hyperref}
\usepackage{graphicx}
\usepackage{listings}
\usepackage{import}
\usepackage{fancyhdr}
\usepackage{tocloft}
\usepackage{adjustbox}

\hypersetup{
    colorlinks=true,
    linktoc=page,
    linkcolor=red,
}

\pagestyle{fancy}
\lhead{Tecnologías específicas de la ingeniería informática}
\chead{}
\cfoot{\thepage}
\rhead{}
\renewcommand{\headrulewidth}{0.2pt}
\renewcommand{\footrulewidth}{0.2pt}
\renewcommand\cftsecleader{\cftdotfill{\cftdotsep}}



\title{Análisis algorítmico: Divide y vencerás}
\author{Siro Sánchez del Amo}
\date{Febrero 2018}

\begin{document}
\maketitle
\vfill 
   {\begin{flushright}
   \includegraphics[width=4cm]{umulogo}
   \end{flushright}}

\thispagestyle{empty}
\newpage

\tableofcontents
\thispagestyle{fancy}
\newpage

\section{Introducción}
Este documento consiste en un análisis algorítmico de la técnicas ''Multiplicación directa'', ''Divide y vencerás'' y ''Karatsuba y Ofman''.

\section{Algoritmo para ''Enteros largos''}
\subsection{Definición del problema}
El problema que se plantea es la multiplicación de números muy largos, lo cual a la larga puede dar problemas ya sea por llegar al límite o por lentitud de las operaciones, para poder realizar operaciones muy largas hemos usado listas y en cada posición de estas habrá un caracter, esto nos dará mas juego que usando simplemente enteros o doble precisión. Con esto en mano, procederemos al análisis de los algoritmos que hemos adaptado al problema.

\subsection{Solución al problema}

\section{Análisis de los algoritmos}
\subsection{Multiplicación directa}
Calculamos la cota inferior:
$$t_m(n)=14+4+5n[Mult Simple]+\sum_{i=0}^{n-1} (2+4+5 n[MultSimple]+12+4 n[Suma]+ \sum_{j=0}^{m} (2))$$
$$t_m(n)=18+5n+18n-18+9n^2-9n+2mn-2m$$
$$t_m(n)=9n^2-14n+2mn-2m$$
$$t_m(n)\in\Omega(n^2)$$
\\
Calculamos cota superior:
$$t_M(n)=5+6+3+7+5n[MultSimple]+ \sum_{i=0}^{n-1}(2+7n+5[MultSimple]+\sum_{j=0}^{m}(2)+17+8n[Suma]+1)+4$$
$$t_M(n)=25+5n-2n-2+7n^27n+5n-5+2mn-2m+17n-17+8n^2-8n+n-1$$
$$t_M(n)=15n^2+15n+2mn-2m$$
$$t_M(n)\in O(n^2)$$
\\
Lo que al final deriva en:
$$
	\left.
        \begin{array}{ll}
           	t_M(n)\in O(n^2) \\
			t_m(n)\in\Omega(n^2)
        \end{array}
    \right\} t_p(n)\in \theta(n^2).
$$
\newpage
\begin{center}
En la siguiente tabla se pueden ver los distintos experimentos realizados: 
\end{center}
\label{md-table}
\begin{adjustbox}{width=0.5\textwidth,center=\textwidth}
\begin{tabular}{|l|l|l|}
\hline
{Dígitos} 					   & {T.experimental} 					   & {Memoria} 						\\ \hline
1                              & $\sim$0.00025s                        & Less than 1KB                  \\ \hline
2                              & $\sim$0.00031s                        & Less than 1KB                  \\ \hline
4                              & $\sim$0.00038s                        & Less than 1KB                  \\ \hline
8                              & $\sim$0.00049s                        & Less than 1KB                  \\ \hline
16                             & $\sim$0.00125s                        & Less than 1KB                  \\ \hline
32                             & $\sim$0.00450s                        & Less than 1KB                  \\ \hline
64                             & $\sim$0.01426s                        & Less than 1KB                  \\ \hline
128                            & $\sim$0.05532s                        & Less than 1KB                  \\ \hline
256                            & $\sim$0.14827s                        & $\sim$33KB                     \\ \hline
512                            & $\sim$0.36122s                        & $\sim$66KB                     \\ \hline
1024                           & $\sim$1.22063s                        & $\sim$132KB                    \\ \hline
\end{tabular}
\end{adjustbox}

\subsection{Divide y vencerás}

\subsection{Karatsuba y Ofman}

\section{Conclusiones experimentales}
\end{document}
